\section{Experiments}
\subsection{Dataset Preparation}
To support personalized, nutritionally aware recipe generation, we defined a "healthy" meal using the following nutritional criteria:

\begin{description}
	\item[Low Glycemic Index (GI)] Ingredients must have a GI $\leq$ 55
	\item[Low Carbohydrates] Total carbohydrates $\leq$ 45--60,g per meal (per adult dietary guidelines)
	\item[Low Added Sugars] Added sugars contribute less than 10\% of total daily caloric intake
	\item[Balanced Macronutrients] Recipes contain adequate protein and healthy fats, and minimal saturated fats
	\item[Portion Control] Servings are appropriately sized to meet average caloric needs
\end{description}

However, we only had time to implement the first criterion, focusing on glycemic index. This choice was made to simplify the dataset preparation process while still allowing for meaningful recipe generation. The glycemic index is a well established measure of how quickly foods raise blood sugar levels, making it a useful proxy for overall meal healthiness.

To construct a dataset aligned with these criteria, I first extracted glycemic index values for base ingredients using data from \cite{foodstruct_glycemic_index}. The RecipeNLG dataset \cite{bien2020recipenlg} was then filtered to retain only those recipes where the majority of ingredients had a glycemic index $\leq$ 55. This process reduced the dataset from 2,231,142 recipes (\texttt{full\_dataset.csv}) to 3,419 recipes (\texttt{h\_recipes\_50pct.csv}).

For efficiency, only recipes with $\leq$ 512 tokens were kept, further narrowing the dataset to 3,398 entries (\texttt{h\_recipes\_50pct\_token\_max512.csv}). This greatly improved training speed and memory usage, particularly given the use of a single 12.7GB GPU (Google Colab environment).

Additional data cleaning was necessary. Recipes with ingredients not listed in the glycemic index were removed. Ingredient duplication (e.g., both “apple” and “apples”) was resolved by adding basic plural checks.

Each recipe was then tagged according to the format required by the RecipeNLG tokenizer:

\begin{description}
	\item[\texttt{<RECIPE\_START>}...\texttt{<RECIPE\_END>}] Encloses the entire recipe
	\item[\texttt{<INPUT\_START>}...\texttt{<NEXT\_INPUT>}...\texttt{<INPUT\_END>}] Tags ingredient names extracted via Named Entity Recognition (NER)
	\item[\texttt{<INGR\_START>}...\texttt{<NEXT\_INGR>}...\texttt{<INGR\_END>}] Tags full ingredient lines
	\item[\texttt{<INSTR\_START>}...\texttt{<NEXT\_INSTR>}...\texttt{<INSTR\_END>}] Tags individual cooking steps
	\item[\texttt{<TITLE\_START>}...\texttt{<TITLE\_END>}] Tags the recipe title
\end{description}

Within each section, items were delimited using \texttt{<NEXT\_INPUT>}, \texttt{<NEXT\_INGR>}, or \texttt{<NEXT\_INSTR>} tags as appropriate. The final preprocessed dataset was saved as \texttt{h\_recipes\_50pct\_token\_max512\_tagged.csv}.

\subsection{Model Training}
The model was fine tuned using the Hugging Face \texttt{Trainer} API \cite{wolf2019huggingface} with the following hyperparameters:

\begin{itemize}
	\item \textbf{Learning Rate:} 5e-5
	\item \textbf{Batch Size:} 8 (per device, for both training and evaluation)
	\item \textbf{Epochs:} 10
	\item \textbf{Optimizer:} AdamW (betas = (0.9, 0.999), epsilon = 1e-8)
	\item \textbf{Scheduler:} Linear learning rate decay with 50 warmup steps
	\item \textbf{Mixed Precision:} Enabled (fp16=True) for faster training on GPU
	\item \textbf{Random Seed:} 42
	\item \textbf{Checkpointing:} Every 50 steps, retaining the two most recent checkpoints
	\item \textbf{Logging:} Every 100 steps (loss and learning rate output to TensorBoard)
\end{itemize}

For the complete training script, see \texttt{notebooks/retrain.ipynb}.

\subsection{Model Run Parameters}
\paragraph{Prompt Format.}

To generate recipes, the model was given prompts in the following structure:

\begin{quote}
	Give me a healthy recipe using the following ingredients (you don't have to use all of them): \texttt{<RECIPE\_START>} \texttt{<INPUT\_START>} chocolate \texttt{<NEXT\_INPUT>} pickles \texttt{<NEXT\_INPUT>} beef \texttt{<NEXT\_INPUT>} pear \texttt{<INPUT\_END>}
\end{quote}

(Whitespace was minimized in practice to reduce token count.)

\paragraph{Tokenization Parameters.}
\begin{itemize}
	\item \texttt{truncation=True}
	\item \texttt{max\_length=512}
	\item \texttt{return\_attention\_mask=True}
\end{itemize}

\paragraph{Generation Parameters.}
\begin{itemize}
	\item \texttt{max\_length=512}
	\item \texttt{num\_beams=2} or \texttt{5} (beam search)
	\item \texttt{no\_repeat\_ngram\_size=2}
	\item \texttt{early\_stopping=True}
	\item \texttt{eos\_token\_id} set to the token ID for \texttt{<RECIPE\_END>}
\end{itemize}

The \texttt{no\_repeat\_ngram\_size=2} setting helped prevent repetitive ingredient or instruction phrases, a common issue in neural text generation. Beam search (with \texttt{num\_beams=2} or \texttt{5}) allowed the model to explore multiple candidate sequences, improving recipe quality.

\subsection{Results}
Below are selected examples generated using the same base ingredients:

\begin{tcolorbox}[recipebox={Melon and Strawberry Ice Cream}]
	\textbf{Base Ingredients:} milk, cream, strawberry, melon, honey

	\vspace{0.5em}
	\textbf{Ingredients:}
	\begin{itemize}
		\item 2 cups milk
		\item 1 cup half-and-half cream
		\item 12 cup strawberry, hulled and quartered
		\item 8 ounces cantaloupe, peeled, seeded, and cut into 1/2-inch cubes
		\item 2 tablespoons honey
	\end{itemize}

	\vspace{0.5em}
	\textbf{Instructions:}
	\begin{enumerate}
		\item In a medium saucepan, combine the milk and cream.
		\item Bring to a boil over medium-high heat, then reduce the heat to low and simmer, stirring occasionally, until the mixture is thickened, about 5 minutes (do not let it boil).
		\item Stir in the strawberries and melon; cover and refrigerate until chilled, at least 2 hours.
	\end{enumerate}
\end{tcolorbox}

\begin{tcolorbox}[recipebox={Melon-Jello Mold}]
	\textbf{Base Ingredients:} milk, cream, strawberry, melon, honey

	\vspace{0.5em}
	\textbf{Ingredients:}
	\begin{itemize}
		\item 1 1/2 cups milk
		\item 1 cup half-and-half cream
		\item 10 ounces strawberry Jell-O gelatin dessert
		\item 8 ounces cantaloupe melon, cubed
		\item 2 tablespoons honey
	\end{itemize}

	\vspace{0.5em}
	\textbf{Instructions:}
	\begin{enumerate}
		\item In a medium saucepan, combine milk and cream. Cook and stir over medium heat until mixture comes to a boil. Remove from heat; stir in gelatin until dissolved. Chill until slightly thickened, about 1 hour.
		\item Stir in melons and honey. Pour into a 6-cup mold coated with cooking spray. Refrigerate until set, at least 4 hours. Unmold onto a serving platter.
	\end{enumerate}
\end{tcolorbox}

Adding an incompatible ingredient like \texttt{salmon} yields a more creative, yet plausible recipe:

\begin{tcolorbox}[recipebox={Grilled Salmon With Strawberry Sauce}]
	\textbf{Base Ingredients:} milk, cream, strawberry, melon, honey, salmon

	\vspace{0.5em}
	\textbf{Ingredients:}
	\begin{itemize}
		\item 1 1/2 cups milk
		\item 1 cup half-and-half cream
		\item 12 ounces strawberry, hulled and quartered
		\item 8 ounces cantaloupes, peeled, seeded and cubed
		\item 2 tablespoons honey
		\item 4 (6-ounce) salmon fillets
	\end{itemize}

	\vspace{0.5em}
	\textbf{Instructions:}
	\begin{enumerate}
		\item In a medium saucepan, combine the milk, half and half, strawberries and melon.
		\item Bring to a boil over medium-high heat, then reduce the heat to medium and simmer until the fruit is soft, about 10 minutes. Stir in the honey and season with salt and pepper. Set aside to cool to room temperature, stirring occasionally to prevent discolouring. Place the salmon in a resealable plastic bag and pour the cooled fruit mixture over the top. Seal the bag, pressing out as much air as possible. Refrigerate for at least 4 hours or overnight, turning once. Remove from the refrigerator 30 minutes before grilling. Preheat an outdoor grill for medium heat (350° to 400°).
		\item Grill salmon, skin side down, until just cooked through, 4 to 6 minutes, depending on thickness. Serve with the strawberry sauce.
	\end{enumerate}
\end{tcolorbox}

However, ingredient pairing is not always optimal. For example:

\begin{tcolorbox}[recipebox={Dried Beef And Pear Salad}]
	\textbf{Base Ingredients:} chocolate, pickles, beef, pear

	\vspace{0.5em}
	\textbf{Ingredients:}
	\begin{itemize}
		\item 1 (4 oz.) pkg. chocolate or butterscotch pudding mix
		\item 1/2 c. chopped pickles
		\item 2 oz. jar dried beef, chopped
		\item peel of 1/4 medium pear
	\end{itemize}

	\vspace{0.5em}
	\textbf{Instructions:}
	\begin{enumerate}
		\item Mix pudding and pickle in a bowl.
		\item Add beef and pear. Chill.
	\end{enumerate}
\end{tcolorbox}
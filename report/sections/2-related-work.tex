\section{Related Work}

Early research in automatic recipe generation centered around multimodal approaches, such as Recipe1M+ \cite{marin2019recipe1m+}, which aligned food images and textual recipes through joint embedding spaces. More recently, the focus has shifted to purely text based generation using large scale language models. Kusupati et al. \cite{lee2020recipegpt} introduced RecipeGPT, a transforme based model trained on a diverse corpus of cooking instructions, demonstrating the feasibility of generating syntactically and semantically coherent recipes.

Health aware recipe generation presents additional challenges. Yang et al. \cite{yang2022nutritional} explored optimizing nutritional content by substituting ingredients while preserving recipe structure. Similarly, Nutri-bot \cite{mehta2023nutribot} integrated user specific dietary profiles and nutritional guidelines to personalize meal recommendations.

In the broader area of diet planning, machine learning has enabled the development of models that optimize meals based on caloric intake, macronutrient balance, and glycemic index \cite{fang2020diet}. These systems typically leverage optimization techniques such as reinforcement learning or graph based search to recommend nutritionally balanced plans.

Large language models (LLMs) have also begun to play a prominent role in health related applications. For example, recent work by Lee et al. \cite{lee2023llmhealth} investigates how LLMs can be used for personalized dietary recommendations, capitalizing on their strength in contextual language understanding and integration of domain knowledge.

Building on these developments, our approach fine tunes GPT-2 on a filtered subset of the RecipeNLG dataset with explicit health related constraints. This bridges flexible text generation with nutritional awareness, enabling the model to produce recipes aligned with user provided ingredients and health goals.

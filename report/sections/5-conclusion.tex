\section{Conclusion}

We have learnt that we could take an existing LLM and fine-tune it with an appropriate dataset to generate healthy recipes. The model was trained on a filtered subset of the RecipeNLG dataset, ensuring that the recipes adhered to specific health guidelines such as low glycemic index and balanced nutrients.

The training process involved careful dataset preparation, including tokenization and tagging of recipe components, which allowed the model to learn the structure and content of healthy recipes effectively. The model was then evaluated using various metrics, demonstrating its ability to generate coherent and relevant recipes based on given ingredients.

However, from the talking with fellow classmates during the presentation, it was clear that the model could be further improved. Some suggestions included:
\begin{itemize}
	\item Incorporating more diverse datasets to enhance the model's understanding of different cuisines and dietary preferences.
	\item Building up on that, adding more specific dietary restrictions, such as vegan or gluten-free, to the training data to allow the model to generate recipes that cater to these needs.
	\item Implementing a feedback loop where users can rate the generated recipes, allowing the model to learn from user preferences and improve over time.
	\item Exploring the use of more advanced LLM architectures or techniques, such as reinforcement learning from human feedback (RLHF), to further enhance the quality and relevance of the generated recipes.
\end{itemize}

Overall, this project has demonstrated the potential of using LLMs for generating healthy recipes, paving the way for future advancements in this area. The model can serve as a valuable tool for individuals seeking to maintain a healthy diet while enjoying diverse and delicious meals.

\subsection{Acknowledgements}
I would like to thank my classmates for their valuable feedback during the presentation, which has helped identify areas for improvement and future work. Their insights have been instrumental in shaping the direction of this project and enhancing its overall quality.

\begin{itemize}
    \item \textbf{Method Design:} David
    \item \textbf{Experiments:} Louis
    \item \textbf{Data Analysis:} Louis
    \item \textbf{Report Writing:} Louis
    \item \textbf{Poster Making:} David
\end{itemize}
